
\documentclass[a4paper,openany,10.5pt]{article}
\usepackage{amsmath}
\usepackage{color}
\usepackage{textcomp}
\usepackage{float}
\usepackage{array}
\usepackage{listings}

\usepackage[final]{graphicx}
\usepackage{subfig}

\usepackage{ragged2e}%Alignment
\usepackage[top=45px, bottom=45px, left=30px, right=30px]{geometry}
%\thispagestyle{empty}  %cancel page number 
\newcommand{\myparagraph}[1]{\paragraph{#1}\mbox{}\\}
\newcommand{\abs}[1]{\left\lvert#1\right\rvert}

\title{CMPT 419 Assignment 2}
\author{Ruijia(Alan) Mao}

\begin{document}
	%Title
	\begin{center}
		\textbf{CMPT 419 Assignment 2 }
	\end{center}
	
	%author
	\begin{flushright}
		\textbf{Name: Ruijia Mao }\\
		\textbf{SFU ID: 301295769}
	\end{flushright}
	
	\section{ }
	The code basically measures $sleep(1)$ using different clock type using function $int\,clock_gettime(clockid_t clk_id, struct 	timespec *tp)$. The difference between these clock type are listed bellow:
	\\ \\
	CLOCK\_REALTIME: System-wide realtime clock. In the code it measures the real time spent to implement the $sleep(1)$, if we change the system time during $sleep(1)$, it will also change.
	\\ \\
	CLOCK\_MONOTONIC: Clock that cannot be set and represents monotonic time since some unspecified starting point. In the code it measures the monotonic real time spent to implement the $sleep(1)$. It is similar the the first clock type except it won't change even if we change the system time.
	\\ \\
	CLOCK\_PROCESS\_CPUTIME\_ID: High-resolution per-process timer from the CPU. This measures the 
	\\ \\
	CLOCK\_THREAD\_CPUTIME\_ID: Thread-specific CPU-time clock.
	\\ \\
	
			
\end{document}